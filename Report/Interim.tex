\documentclass[10pt]{ecsprogressreport}
\begin{document}

\title{Killing the Kill Cord}
\authors{Reece Norton James Williams}
\supervisor{Dr Alex Wheddell}
\examiner{-}
\degree {MEng Electronic Engineering}
\maketitle

\begin{abstract}
This project researches into sensor systems to ensure safety in the event of dangerous conditions exerted in power boating. In the past, existing methods have been ineffective. Currently, the common method is for the operator to attach a cord between themselves and the controls, and when this link is broken due to any circumstance, the engine is stalled. This relies heavily on the operator remembering to attach the device to both themselves and the controls, leading to catastrophic accidents due to human error. The goal of this project is to eliminate human error by creating a system which monitors sensors within the helm, allowing reliable and correct actions to be taken based on the feedback. Each scenario is tethered for allowing for the optimum boating experience, whilst maintaining the safety. A prototype system was designed and manufactured.
\end{abstract}

\tableofcontents

\section{Introduction}

\section{Research}

\subsection{Existing Devices}

\subsection{Accidents}

\subsection{Regulations}

\subsection{Sensors}

\subsubsection{Wheel}

\subsubsection*{Capactive}

\subsubsection*{Resistive}

\subsubsection{Throttle}

\subsubsection{Proximity}

\subsubsection*{Infra-red}

\subsubsection*{Ultrasound}

\subsection{Scenarios}

\section{Work Completed}

\subsection{Testing}

\section{Work Remaining}




\end{document}